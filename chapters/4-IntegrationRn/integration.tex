 \begin{motivating}
    What is the definition of the integral in single variable calculus? 
    \end{motivating}
    

\section{Defining the integral}

\begin{motivating}
    What is the definition of the integral in single variable calculus? 
\end{motivating}

PICTURE

    \begin{enumerate}
        \item Can we define the integral such that we can deal with both  $\int_\infty^\infty f(x) \ dx$ and $\int_a^b f(x) \ dx$?
        \item Can we define the integral in a way that easily generalizes to multivariable functions $f: \R^n \to \R$?
        \item What kinds of functions can we integrate? 
    \end{enumerate}

In these lecture notes, we'll be dealing with proper/definite integrals.  That is, we will want our integrals to have finite values.\footnote{It takes some care to define non-proper integrals in multivariable calculus!}

That is, we'll be working with functions that are \textbf{bounded} with \textbf{bounded support}.

\begin{definition}
A subset $D \subset \R^n$ is \textbf{bounded} if there exists some $r > 0$ such that $D \subset B_r(\bm{0})$.
\end{definition}

\fixthis{recall}
    
    
    \begin{definition}
    A function $f: A \subset \R^n \to \R$ is \textbf{bounded}\define{bounded function} if its image $$\{f(\bm{x}) \ | \ \bm{x} \in A \}$$ is a bounded subset of $\R$.
    \end{definition}

\fixthis{recall}

\begin{definition}
A point $\bm{x} \in \R^n$ is a \textbf{boundary point} of $D \subset \R^n$ if: for all $\varepsilon > 0$,

\begin{enumerate}
    \item $B_\varepsilon(\bm{x}) \cap D$ is non-empty, \textbf{and}
    \item $B_\varepsilon(\bm{x}) \cap D^{c}$ is non-empty.
\end{enumerate}
\end{definition}
    
    \begin{definition}
    The \textbf{closure} of a set $D \subset \R^n$ is the union of $D$ and the boundary of $D$.  That is, the closure is the set $$\overline{D} = \{\bm{x} \in \R^n \ | \ B_r(\bm{x}) \cap D \neq \varnothing\}$$  
    \end{definition}

\begin{proposition}
    A set $D$ is closed (\fixthis{recall}) if and only if $$D = \overline{D}$$
\end{proposition}

\begin{definition}
    The \textbf{support}\define{support of a function} of a function $f : A \subset \R^n \to \R$ is the \textbf{closure} of the set $$\{\bm{x} \in A \ | \ f(\bm{x}) \neq 0\}$$
\end{definition}

\begin{definition}
    A function has \textbf{bounded support}\define{bounded support} if its support is bounded.  

    Equivalently, there exists $R>0$ such that $f(\bm{x}) = 0$ for all $||\bm{x}|| > R$.
    \end{definition}

We can now define the notion of multivariable integral for bounded functions with bounded support.  


\subsection{Integration over $\R^n$}

The main idea of integration is not so difficult: 

\begin{enumerate}
    \item We chop up our region into pieces
    \item We sample the function on each of the pieces, and sum over the pieces.
    \item We take the limit of the sum as the size of the pieces shrinks.
\end{enumerate}

However, we will need to be careful, as we will encounter some technical difficulties in making these ideas rigorous.

We will begin with chopping up $\R^n$ into pieces.  In technical terms, we will construct a partition of $\R^n$:

\begin{definition}
    Given a vector $\bm{k} = \langle k_1, \cdots, k_n\rangle \in \Z^n \subset \R^n$ (that is, $k_i \in \Z$ for all $i$), we can define the \textbf{dyadic cube} $C_{\bm{k},N}$ in $\R^n$ as
    $$C_{\bm{k},N} := \left\{ \bm{x} = \langle x_1, \cdots, x_n\rangle \in \R^n \ | \ \frac{k_i}{2^N} \leq x_i < \frac{k_i+1}{2^N}\right\}$$

\fixthis{picture}
    
    \end{definition}

\begin{proposition}
    The volume of a dyadic cube $C_{\bm{k},N}$ in $\R^n$ is $\frac{1}{2^{Nn}}$.
    \end{proposition}

\begin{proposition}
    For a fixed $N$, the collection of all dyadic cubes $$D_N(\bm{\R^n}) := \left\{C_{\bm{k},N} \ | \ \textnormal{for all} \ \bm{k} \in \Z^n \right\}$$ partitions $\R^n$. We call $D_N(\bm{\R^n})$ the $N$\textbf{-th dyadic partition of} $\R^n$.
    \end{proposition}

The dyadic cubes is a choice of \textbf{partition} of $\R^n$ - there are many other ways to partition $\R^n$, and they each lead to an equivalent definition of the multivariable integral.  For example, it turns out that the partitions do not need to be uniform (e.g. we could take rectangles with differing widths).  Moreover, we could take partitions depending on the range of the function (this leads to the notion of the Lebesgue integral).

We have chosen this particular partition as it is easy to define. 


Given our dyadic cube partition, we now turn to sampling points from the partition:

\fixthis{lower left corner, midpoint, etc.}

\begin{definition}\label{def:riemannsum}
    Given a choice of $\bm{x_{k,N}} \in C_{\bm{k},N}$ for every cube, the $N$-th \textbf{Riemann sum}\define{Riemann sum} is defined as 
    $$R_N(f) := \sum_{C \in D_N(\R^n)} f(\bm{x_{k,N}}) \ \textnormal{vol(C)}$$
    \end{definition}

We have many choices for sampling points, and it is a priori unclear that these various sampling methods should produce the same value for our integral when we take the limit as $N$ goes to $\infty$.

Thus, we will use Darboux sums to define the multivariable integral - this will guarantee that if the limit as $N$ goes to $\infty$ exists, then any sampling method will produce the same value\footnote{See proposition \ref{prop:riemannisintegrable} and exercise \ref{problem:riemannintegrabletodarboux}}.

 \begin{definition}
    Let $X \subset \R$.  A number $a \in \R$ is an \textbf{upper bound} if for every $x \in X$, we have that $x \leq a$.  
    
    A number $b \in \R$ is the \textbf{supremum of $X$} (or \textbf{least upper bound of $X$}) if for all upper bounds $a$ of $X$, we have that $b \leq a$.
    
    We write $b := \sup(X)$.  If $X$ is not bounded above, we write $\sup(X) = \infty$.
    
    \end{definition}

\begin{definition}
    Let $X \subset \R$.  A number $a \in \R$ is an \textbf{lower bound} if for every $x \in X$, we have that $x \geq a$.  
    
    A number $b \in \R$ is the \textbf{infimum of $X$} (or \textbf{greatest lower bound of $X$}) if for all lower bounds $a$ of $X$, we have that $b \geq a$.
    
    We write $b := \inf(X)$.  If $X$ is not bounded below, we write $\inf(X) = -\infty$.
    
    \end{definition}

    \begin{theorem}[Completeness of $\R$]
    Every nonempty subset $X \subset \R$ has a supremum and infimum. Moreover, $\sup(X)$ and $\inf(X)$ are unique.
    \end{theorem}

    \begin{definition}
    Let $f: \R^n \to \R$ be a function, and $D \subset \R^n$ an arbitrary subset. We will consider the following quantities:
    $$M_D(f) := \sup(\{f(\bm{x}) \ | \ \bm{x} \in D\}) \qquad m_D(f) := \inf(\{f(\bm{x}) \ | \ \bm{x} \in D\})$$
    
    \end{definition}
    
    \begin{definition}
    Let $f: \R^n \to \R$ be a bounded function with bounded support. The $N$\textbf{-th upper Darboux sum} and $N$\textbf{-th lower Darboux sum} $f$ are defined as 
    $$U_N(f) := \sum_{C \in D_N(\R^n)} M_C(f)\textnormal{vol(C)} \qquad L_N(f) := \sum_{C \in D_N(\R^n)} m_C(f)\textnormal{vol(C)}$$
    
    \end{definition}

    \begin{proposition}
    $$U_N(f) := \frac{1}{2^{Nn}} \sum_{C \in D_N(\R^n)} M_C(f)  \qquad L_N(f) := \frac{1}{2^{Nn}}\sum_{C \in D_N(\R^n)} m_C(f)$$
    \end{proposition}
    
    \begin{proposition}
    As $N$ increases, $U_N(f)$ decreases and $L_N(f)$ increases.
    \end{proposition}

    \fixthis{picture}
    
    \begin{definition}
    Let $f: \R^n \to \R$ be a bounded function with bounded support.  The \textbf{upper Darboux integral} and \textbf{lower Darboux integral} of $f$ are defined as
    $$U(f) := \lim_{N \to \infty} U_N(f)   \qquad L(f) := \lim_{N \to \infty} L_N(f)$$
    \end{definition}

    \begin{remark}
        \fixthis{limits exist}
    \end{remark}

\begin{definition}\label{def:integrable}
    Let $f: \R^n \to \R$ be a bounded function with bounded support.
    We say that $f$ is \textbf{integrable}\define{integrable} if $U(f) = L(f)$.
    
    The \textbf{integral} of $f$ is $$\int_{\R^n} f(\bm{x}) \ dV := U(f) = L(f)$$

    Equivalently \fixthis{ref}, for any $\varepsilon > 0$,
    there exists $N$ such that $|U_N(f) - L_N(f)| < \varepsilon$
    \end{definition}

    

    
    \begin{corollary}
    If $f$ is integrable, we can use $U_N(f)$ or $L_N(f)$ to estimate $\int_{\R^n} f(\bm{x}) \ dV$. 
    \end{corollary}

    \begin{example}
        
    \begin{theorem}
     If $f : \R^n \to \R$ is a continuous function that is bounded with bounded support, then $f$ is integrable.   
    \end{theorem}
    \end{example}

    
However, there are some functions that are ``obviously" integrable, yet are not continuous:

\begin{example}
    The indicator function on the unit disk
\end{example}
    
    In section \ref{sec:integrability}, we will discuss criteria for a function $f$ to be integrable.  For now, whenever we say integrable, you should think of continuous functions that are bounded with bounded support.



\begin{remark}
Note that if the integral $\int_{\R^n} f \ dV$ exists, then these two boundedness conditions($f$ is bounded and $f$ has bounded support) guarantee that the integral is finite.    
\end{remark}


\begin{remark}
    Using the Darboux formulation, it follows that if $f$ is integrable, then any choice of sampling  \fixthis{will result in the same value for the integral}
\end{remark}



\begin{proposition}\label{prop:riemannisintegrable}
    If $f: \R^n \to \R$ is integrable, then the integral is the limit of the Riemann sum\footnote{See definition \ref{def:riemannsum}}.  That is,
    $$\lim_{N \to \infty} R_N(f) = \int_{\R^n} f(\bm{x}) \ dV$$
    \end{proposition}
    
    \begin{example}
     \textbf{Warning!} This only works if you know $f$ is integrable.  The Riemann sum may converge even if the function is not integrable.  Consider the function: $$f(x) := \left\{
		\begin{array}{ll}
			1 & \text{ if } x \in [0,1] \text{ and }  x \notin \Q \\
			0 & \text{ otherwise}
		\end{array}
		\right.$$   
    \end{example}


\subsection{Integration over subsets of $\R^n$}

    \begin{motivating}
        How do we integrate over sub-regions of $\R^n$?
    \end{motivating}

    We have defined integration for integrable functions $f : \R^n \to \R$ over all of $\R^n$. However, we should be able to integrate over subregions (\fixthis{example})

    For example, we know how to calculate $\int_0^1 x \ dx$.


    

    Suppose that $B \subset \R^n$ is a region in $\R^n$.  We have two potential ways to define the integral of $f$ over $B$ (denoted $\int_B f \ dV$).  We can either:

    \begin{enumerate}
        \item incorporate $B$ into the definition of the integral $\int_B f \ dV$.        
    In other words, given any subset $B$, we would need to partition $B$ into smaller subsets, and take a limit.
        \item or we can treat $B$ as part of the \textbf{data} of the function.
        That is, we can use the partition of $\R^n$ that we constructed previously to yield a partition of $B$.
        
    \end{enumerate}

    We will take the second perspective, and define the integral of $f$ over $B$ in terms of some integral $\int_{\R^n} g \ dV$.  We can do so using \textbf{indicator functions}:

    \fixthis{ref indicator function}

    \begin{definition}
    Let $B \subset \R^n$ be a subset.  The \textbf{indicator function} $1_B$ is the function 
    $$1_B(\bm{x}) := \left\{
		\begin{array}{ll}
			1 & \text{ if } \bm{x} \in B \\
			0 & \text{ if } \bm{x} \notin B
		\end{array}
		\right.$$
    
    \end{definition}

    Observe that given a function $f: \R^n \to \R$, then the function $f(\bm{x})1_B(\bm{x})$ is the piecewise function
    $$f(\bm{x})1_B(\bm{x}) := \left\{
		\begin{array}{ll}
			f(\bm{x}) & \text{ if } \bm{x} \in B \\
			0 & \text{ if } \bm{x} \notin B
		\end{array}
		\right.$$

    This construction restricts the support of $f$ to the region $B$.  Thus, to integrate a function $f: \R^n \to \R$ over $B$, it is equivalent to integrate the function $f(\bm{x})1_B(\bm{x})$ instead.

    \begin{definition}
     Let $B \subset \R^n$, and let $f : \R^n \to \R$ be an integrable function.  Then we can define the \textbf{integral of $f$ over the region $B$} (denoted $\int_B f(\bm{x}) \ dV$ )   as

    $$\int_B f(\bm{x}) \ dV := \int_{\R^n} f(\bm{x})1_B(\bm{x}) \ dV$$
     
    \end{definition}
    
    \fixthis{picture}

    \begin{definition}
    Let $A \subset \R^n$.  If $1_A : \R^n \to \R$ is integrable, then the $n$\textbf{-dimensional volume} of $A$ is $$\textnormal{vol}_n(A) := \int_{\R^n} 1_A \ dV$$
    \end{definition}





    Moreover, we can also use indicator functions to integrate functions that are not defined on all of $\R^n$!  For example, we know how to integrate $f(x) = \sqrt{x}$
    
    \fixthis{example}

    Let $A \subset \R^n$, and let $f : A \to \R$ be a function. We can extend $f$ to a function $\Tilde{f}(\bm{x}) : \R^n \to \R$ by defining
    $$\Tilde{f}(\bm{x}) := \left\{
		\begin{array}{ll}
			f(\bm{x}) & \text{ if } \bm{x} \in A \\
			0 & \text{ if } \bm{x} \notin A
		\end{array}
		\right.$$
        
    We will often use the following abusive notation when we want to indicate the domain $A$:
    $$f(\bm{x})1_A(\bm{x}) := \Tilde{f}(\bm{x})$$

    \begin{definition}
    A function $f: A \subset \R^n \to \R$ is \textbf{integrable} if the function $f(\bm{x})1_A(\bm{x})$ is integrable, and we can define the integral of $f$ (denoted $\int_A f(\bm{x}) \ dV$) as

    $$\int_A f(\bm{x}) \ dV := \int_{\R^n} f(\bm{x})1_A(\bm{x}) \ dV$$
    
    \end{definition}



    We can combine these two notions as well to define integrals of integrable functions $f : A \subset \R^n \to \R$ over regions $B \subset \R^n$.

    \begin{definition}
        Let $B \subset \R^n$, and let $f : A \subset \R^n \to \R$ be an integrable function. Then we can define the integral $\int_B f(\bm{x}) \ dV$ as

        $$\int_B f(\bm{x}) \ dV := \int_{\R^n} f(\bm{x})1_A(\bm{x})1_B(\bm{x}) \ dV$$
        
        By construction,

        \begin{align*}
        \int_B f(\bm{x}) \ dV &= \int_B \Tilde{f}(\bm{x}) \ dV \\
        &= \int_{\R^n} \Tilde{f}(\bm{x})1_B(\bm{x}) \ dV \\
        &= \int_{\R^n} f(\bm{x})1_A(\bm{x})1_B(\bm{x}) \ dV
    \end{align*}
    \end{definition}

\subsection{Properties of the integral}

In this section, we will prove some properties about the multivariable integral.  Recall that when you see integrability, you should think about continuous functions that are bounded with bounded support.

\begin{theorem}
    
    Let $f, g : \R^n \to \R$ be two integrable functions.  Then
    
    \begin{enumerate}
        \item $f+g$ is also integrable, and
        $$\int_{\R^n} f+g \ dV = \int_{\R^n} f \ dV + \int_{\R^n} g \ dV$$
        \item If $\lambda \in \R$, then $\lambda f$ is integrable, and 
        $$\int_{\R^n} \lambda f \ dV = \lambda \int_{\R^n} f \ dV$$
        
    \end{enumerate}
    \end{theorem}

    \begin{theorem}
    
    Let $f, g : \R^n \to \R$ be two integrable functions.  Then
    
    \begin{enumerate}
        \item If $f(\bm{x}) \leq g(\bm{x})$ for all $\bm{x} \in \R^n$, then $$\int_{\R^n} f \ dV \leq \int_{\R^n} g \ dV$$
        \item $|f|(\bm{x}) := |f(\bm{x})|$ is integrable, and $$\left|\int_{\R^n} f \ dV\right| \leq \int_{\R^n} |f| \ dV$$
    \end{enumerate}
    
    \end{theorem}

    \begin{corollary}
    We can reduce to studying integrals of non-negative functions.
    \end{corollary}
    
    \begin{definition}
    Given a function $f : \R^n \to \R$, we define two auxillary functions, $f^+$ and $f^-$.
    $$f^+(\bm{x}) := \left\{
		\begin{array}{ll}
			f(\bm{x}) & \text{ if } f(\bm{x}) \geq 0 \\
			0 & \text{ otherwise}
		\end{array}
		\right. \qquad f^-(\bm{x}) := \left\{
		\begin{array}{ll}
			-f(\bm{x}) & \text{ if } f(\bm{x}) \leq 0 \\
			0 & \text{ otherwise}
		\end{array}
		\right.$$
    
    \end{definition}
    
    \begin{proposition}
    $f(\bm{x}) = f^+(\bm{x}) - f^-(\bm{x})$.
    \end{proposition}


    \begin{proposition}
    Suppose that $f(\bm{x})$ is integrable on $\R^n$, and $g(\bm{y})$ is integrable on $\R^m$.  Then $h(\bm{x},\bm{y}) = f(\bm{x})g(\bm{y})$ is integrable on $\R^{n+m}$, and 
    $$\int_{\R^{n+m}} h \ dV \ dW = \int_{\R^n} f \ dV \int_{\R^n} g \ dW$$
    \end{proposition}
    
    \begin{remark}
        \textbf{Warning:} Note that $\bm{x}$ and $\bm{y}$ are necessarily different variables.  Furthermore, $$\int_{\R^{n}} f(\bm{x})g(\bm{x}) \ dV \neq \int_{\R^n} f(\bm{x}) \ dV \int_{\R^n} g(\bm{x}) \ dV$$
    \end{remark}

    \begin{example}
        \fixthis{example}
    \end{example}
    
\subsection{Exercises}

\begin{problem}{examplesboundedboundedsupport}
Come up with examples of
       \begin{enumerate}
           \item a multivariable function that is not bounded
           \item a multivariable function that does not have bounded support
           \item A multivariable function that is neither bounded nor has compact support.
       \end{enumerate}
\end{problem}

\begin{problem}{boundedboundedsupportvspace}
    Prove that if $f$ and $g$ are functions that are bounded with bounded support, then $f+g$ and $\lambda f$ are also functions that are bounded with bounded support.
\end{problem}

\begin{problem}{boundsdarboux}
    Prove that $U_N(f) \geq L_N(f)$.
\end{problem}

\begin{problem}{upperdarbouxexist}
    Prove that $U(f) := \lim_{N \to \infty} U_N(f)$ exists \fixthis{reference problem}
\end{problem}

\begin{problem}{lowerdarbouxexist}
    Prove that $L(f) := \lim_{N \to \infty} L_N(f)$ exists \fixthis{reference problem}
\end{problem}

\begin{problem}{riemannintegrabletodarboux}
    Let $\bm{R_{k,N}} \in C_{\bm{k},N}$ be an arbitrary sampling of the dyadic partition, and define 
    $$R_N(f) := \sum_{C \in D_N(\R^n)} f(\bm{R_{k,N}}) \textnormal{vol(C)}$$
    
    Prove that if $f$ is integrable, then
    $$\lim_{N\to\infty} R_N(f) = \int_{\R^n}f \ dV $$

    (\textbf{Hint:} use the squeeze theorem!)
\end{problem}









\section{Calculating multivariable integrals}

    \begin{motivating}
        Given a region $B \subset \R^n$ and an integrable function $f$, how can we compute $\int_B f(\bm{x}) \ dV$?
    \end{motivating}

    \begin{theorem}[Fubini]
    Let $f(\bm{x}): \R^n \to \R$ be a continuous function that is bounded with bounded support, and let $(i_1, \cdots i_n)$ be a permutation of the set $\{1, \cdots, n\}$. Then 
    
    $$\int_{\R^n} f(\bm{x}) \ dV = \int_{-\infty}^{\infty} \left( \cdots \left(\int_{-\infty}^{\infty} f(\bm{x}) \ dx_{i_1} \right) \cdots \right) \ dx_{i_n}$$

    \end{theorem}

    That is, we can compute an integral $\int_{\R^n} f(\bm{x}) \ dV$ as an iterated integral, in any variable order!

    \begin{example}
        Let $R$ be the region $[0,1] \times [0,1]$.  Compute the integral 
        $$\iint_R xe^{xy} \ dA $$
        By Fubini's theorem, there are two iterated integrals we can use.  Which one is easier?
    \end{example}


    \begin{remark}
        \fixthis{Relationship to Clairaut's theorem}
    \end{remark}

    \begin{example}
    Calculate the integral of $f(x,y,z) = xyz$ over the region $W$ in the first octant bounded by $z = 4 - y^2$, $y = 2x$, $z = 0$, and $x= 0$.
    \end{example}

    \begin{example}
        Compute the integral of $f(x,y) = \frac{1}{\ln(x)}$ over the domain $D$ bounded by $x = e^y$ and $x= e^{\sqrt{y}}$.
    \end{example}

\begin{theorem}[Decomposition of Domains]
        Let $K$ be a compact subset in $\R^n$ such that its boundary $\partial D$ has volume zero.  Furthermore, let $K = K_1 \cup K_2$, such that $K_1$ and $K_2$ are compact, and the intersection $K_1 \cap K_2$ has volume 0. 
    
    
    Let $f : K \to \R^n$ be a continuous function.  Then $f$ is integrable over $K_1$ and $K_2$, and 
    $$\int_K f(\bm{x}) \ dA = \int_{K_1} f(\bm{x}) \ dA + \int_{K_2} f(\bm{x}) \ dA $$ 
    \end{theorem}














\section{Integrability}\label{sec:integrability}

In the previous sections, we have defined the notion of integrability, which we will restate below:

\begin{definition}
    Let $f: \R^n \to \R$ be a bounded function with bounded support.
    We say that $f$ is \textbf{integrable} if $U(f) = L(f)$.
    
    The \textbf{integral} of $f$ is $$\int_{\R^n} f(\bm{x}) \ dV := U(f) = L(f)$$

    Equivalently \fixthis{ref}, for any $\varepsilon > 0$,
    there exists $N$ such that $|U_N(f) - L_N(f)| < \varepsilon$
    \end{definition}

The examples we've had so far are the following:
        
    \begin{theorem}
     If $f : \R^n \to \R$ is a continuous function that is bounded with bounded support, then $f$ is integrable.   
    \end{theorem}


    \begin{definition}
    Let $f: \R^n \to \R$ be a function, and let $A \subset \R^n$.  The \textbf{oscillation} of $f$ over $A$ is defined as $$\text{osc}_A(f) := M_A(f) - m_A(f)$$
    \end{definition}
    
    \begin{theorem}
    A function $f: \R^n \to \R$ is integrable if and only if
    \begin{enumerate}
        \item $f$ is bounded with bounded support
        \item For all $\varepsilon > 0$, there exists $N$ such that $$\sum_{\{C \in D_N \ | \ \text{osc}_C(f) > \varepsilon \}} \textnormal{vol}_n C < \varepsilon$$
    \end{enumerate}
    \end{theorem}

\begin{proof}
    fixthis
\end{proof}

\begin{example}
    The indicator function on the unit disk
\end{example}

\begin{example}

The function $\sin(\frac{1}{x})$
    
%     \begin{center}
    
%     \begin{tikzpicture}[x=8cm]
%     \draw[xstep=.2,ystep=.5,lightgray,ultra thin] (-0.1,-1.5) grid (1.1,1.5);
%     \draw[->] (0,0) -- (1.1,0) node[right] {$x$};
%     \draw[->] (0,-1) -- (0,1.1) node[above] {$y$};
%   \draw[blue,domain=0.01:1,samples=5000] plot (\x, {sin((1/\x)r)});
% \end{tikzpicture}    
%     \end{center}
    
\end{example}

















\subsection{Exercises}

\begin{problem}{integrabilitycounterexamples1}
Come up with (and prove) an example of a function that is not integrable.
\end{problem}

\begin{problem}{integrabilitycounterexamples2}
Come up with (and prove) an example of a function where $|f|$ is integrable, but $f$ is not integrable.
\end{problem}

\section{Change of variables}

\subsection{Polar, cylindrcial, and spherical coordinate}

\subsection{General Change of Variables}

\fixthis{recall Jacobian}




\begin{theorem}[The change of variables formula in $\R^2$]
    Let  $G(u,v) = \left(x(u,v), y(u,v)\right)$ be a map such that
    
    \begin{enumerate}
        \item $G$ sends a region $D_0$ to $D$.  
        \item $G$ is a $C^1$ mapping.
        \item $G$ is one-to-one on the interior of $D_0$.  
    \end{enumerate}

    
    If $f(x,y)$ is continuous, then
    $$\iint_D f(x,y) \ dx \ dy = \iint_{D_0} f(x(u,v), y(u,v)) \left|\textnormal{Jac}(G)\right| \ du \ dv$$
    
\end{theorem}

\begin{example}
    Polar coordinates map defined by $$G(r,\theta) = (r\cos(\theta), r\sin(\theta))$$
\end{example}

\begin{remark}
    Sometimes, it is easier to find a map going in the \textit{wrong direction}, $$F(x,y) = (u(x,y),v(x,y))$$
    Then $G = F^{-1}$.
    \end{remark}
    
        \begin{figure}
        \centering
        \includegraphics[scale=0.45]{pictures/map.png}
    \end{figure}
    
    \begin{theorem}
    If $G = F^{-1}$, and $\textnormal{Jac}(F) \neq 0$, then $\textnormal{Jac}(G) = \frac{1}{\textnormal{Jac}(F)}$
    \end{theorem}










    \begin{theorem}[Change of variables]
    Let $K \subset \R^n$ be a compact set such that $\text{vol}_n(\partial K) = 0$.  Let $U \subset \R^n$ be an open set containing $K$. Let $$\Phi : U \to \R^n$$ be a map such that 
    \begin{enumerate}
        \item $\Phi$ is a $C^1$ mapping.
        \item $\Phi$ is injective on the interior of $K$.  
        \item $\det [D(\Phi)] \neq 0$ on the interior of $K$.  
    \end{enumerate}
    Then if $f: \Phi(K) \to \R$ is a continuous function, then 
    $$\int_{\Phi(K)} f \  dV = \int_K(f \circ \Phi) \cdot |\det [D(\Phi)]| \ dV$$
    \end{theorem}

    \fixthis{examples that show necessity}