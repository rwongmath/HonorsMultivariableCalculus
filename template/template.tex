\documentclass{wongtreebook}
\Title{Honors Multivariable Calculus}
\Author{Richard Wong}
%\\Colophon{Released under the \LaTeX{} Public Project License 1.3c}
\BibliographyFile{willowtreebook}
\usepackage{fancyvrb}
\usepackage{inconsolata}
\begin{document}
%\includegraphics[width=\textwidth]{willow-2.jpg}%
\chapter{Preface}
The Willow Tree Book class is a simplified derivative of the \verb!memoir! book class.
I use it for my lecture notes.
The document you are reading is in the Willow Tree Book class.
\afterpreface


\chapter{Use}
Here is a typical book using the Willow Tree Book class:
\begin{Verbatim}[frame=leftline]
\documentclass{willowtreebook}
\Title{Odyssey}
\Author{Homer}
\BibliographyFile{odyssey} 		
	% The name of the .bib file, without file extension.
\begin{document}


\chapter{Preface}
This is the preface to my book.
\afterpreface
\chapter{We meet Odysseus}
Tell me, O muse, of that ingenious hero
    % The rest of the text goes here.
Thus spoke Minerva, and Ulysses obeyed her gladly. Then Minerva assumed
the form and voice of Mentor, and presently made a covenant of peace
between the two contending parties.
\par\bigskip\noindent
THE END
	% End the document without loading the bibliography
	% or the index, or the list of notation.
\end{document}
\end{Verbatim}
Compile with \verb!latex! or \verb!pdflatex!.









\chapter{Definitions, Problems and Theorems}
\section{Definitions}
We can define a term like \emph{hamster}\define{hamster}, or say that the term hamster\SubIndex{hamster} appears again later.
\begin{Verbatim}[frame=leftline]
\chapter{Definitions, Problems and Theorems}
\section{Definitions}
We can define a term like \emph{hamster}\define{hamster}, 
or say that the term hamster\SubIndex{hamster} appears again later.
\end{Verbatim}
Compile, for a book called \verb!filename.tex!, with
\begin{Verbatim}[frame=leftline]
	makeindex filename
\end{Verbatim}
We add notation like when we use a variable called \(\omega\),
we put it in the list of notation.%
\Notation{omega}{\omega}{A variable called $\omega$}
\begin{Verbatim}
We add notation like when we use a variable called \(\omega\),
we put it in the list of notation.%
\Notation{omega}{\omega}{A variable called $\omega$}
\end{Verbatim}
If you use notation, compile with
\begin{Verbatim}[frame=leftline]
	makeindex -s notation.gst -o not.gls not.glo  
\end{Verbatim}

\section{Theorems}
You have the usual theorem environments, like \texttt{amsthm}.
\begin{theorem}[Pythagoras]
In any triangle with sides of lengths \(a,b,c\), \(a^2+b^2=c^2\) just when the angle opposite the side of length \(c\) is a right angle.
\end{theorem}
\begin{Verbatim}[frame=leftline]
\begin{theorem}[Pythagoras]
In any triangle with sides of lengths \(a,b,c\), 
\(a^2+b^2=c^2\) just when the angle opposite the 
side of length \(c\) is a right angle.
\end{theorem}
\end{Verbatim}





\section{Problems}
\begin{problem}{label.for.the.first.problem}
What is the point of your life?
\end{problem}
\begin{answer}{label.for.the.first.problem}
Your life is pointless.
\end{answer}
In problem~\ref{problem:label.for.the.first.problem}, we can clearly see ...
\begin{Verbatim}[frame=leftline]
		% We add problems by:
\begin{problem}{label.for.the.first.problem}
What is the point of your life?
\end{problem}
		% and answers by:
\begin{answer}{label.for.the.first.problem}
Your life is pointless.
\end{answer}
In problem~\ref{problem:label.for.the.first.problem}, we can clearly see ...
\end{Verbatim}
\chapter{Citations}
Our bibliography file looks like
\VerbatimInput[frame=leftline]{willowtreebook.bib}
We can cite works from the bibliography, like Homer~\cite{Homer:Iliad}, p. 12.
\begin{Verbatim}[frame=leftline]
We can cite works from the bibliography, like Homer~\cite{Homer:Iliad}, p. 12.
\end{Verbatim}
Compile with \verb!bibtex!.




\newpage
\section{Examples}
I often want to present an example, and make clear where it starts and stops.
\begin{example}
The integral 
\[
\int e^{x^2}x\,dx
\]
is evaluated by substituting \(u=x^2\), so
\[
\int e^{x^2}x\,dx=\int e^u \frac{du}{2}.
\]
\end{example}
\begin{Verbatim}[frame=leftline]
\begin{example}
The integral 
\[
\int e^{x^2}x\,dx
\]
is evaluated by substituting \(u=x^2\), so
\[
\int e^{x^2}x\,dx=\int e^u \frac{du}{2}.
\]
\end{example}
\end{Verbatim}






\section{Preambles}
We can put some \LaTeX{} code before the hints:
\begin{Verbatim}[frame=leftline]
\RenewDocumentCommand\hintsPreamble{}{
\par\noindent{}
\textit{When you are describing, \\ 
A shape, or sound, or tint; \\ 
Don't state the matter plainly, \\ 
But put it in a hint; \\
And learn to look at all things, \\ 
With a sort of mental squint.}
\par\noindent{}---\ {Lewis Carroll}}
\end{Verbatim}
\RenewDocumentCommand\hintsPreamble{}{\par\noindent{}\textit{When you are describing, \\ A shape, or sound, or tint; \\ Don't state the matter plainly, \\ But put it in a hint; \\\ And learn to look at all things, \\ With a sort of mental squint.}\par\noindent{}---\ {Lewis Carroll}}
or before the bibliography:
\begin{Verbatim}[frame=leftline]
\RenewDocumentCommand\bibliographyPreamble{}{
\par\noindent
\textit{If those books are in agreement with the Quran, 
we have no need of them; 
and if these are opposed to the Quran, 
destroy them.}
\par\noindent{}---\ {Omar}}
\end{Verbatim}
\RenewDocumentCommand\bibliographyPreamble{}{\par\noindent\textit{If those books are in agreement with the Quran, we have no need of them; and if these are opposed to the Quran, destroy them.}\par\noindent{}---\ {Omar}}
\end{document}